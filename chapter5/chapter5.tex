\documentclass[a4paper,11pt,oneside]{article}

\usepackage{amsmath,amssymb,epsfig}
\usepackage[T1]{fontenc}
\usepackage{ae,aecompl}
\usepackage{url}
\usepackage{subfigure}
\addtolength{\voffset}{-1cm}
\addtolength{\hoffset}{-1cm}
\setlength{\parindent}{0in}
\addtolength{\textwidth}{1.8cm}
\addtolength{\textheight}{1cm}
\addtolength{\parskip}{.5cm}


% Example definitions.
% --------------------
\def\x{{\mathbf x}}
\def\X{{\mathbf X}}
\def\u{{\mathbf u}}
\def\U{{\mathbf U}}
\def\x{{\mathbf x}}
\def\s{{\mathbf s}}
\def\A{{\mathbf A}}
\def\y{{\mathbf y}}
\def\W{{\mathbf W}}
\def\w{{\mathbf w}}
\def\B{{\mathbf B}}
\def\D{{\mathbf D}}
\def\a{{\mathbf a}}
\def\D{{\mathbf D}}
\def\P{{\mathbf P}}
\def\n{{\mathbf n}}
\def\V{{\mathbf V}}
\def\R{{\mathbf R}}
\def\I{{\mathbf I}}
\def\M{{\mathbf M}}
\def\sech{{\mathrm{sech}}}
\def\L{{\cal L}}
\def\Cum{{\rm{Cum}}}
\def\var{{\rm{var}}}
\def\T{{\mathbf T}}
\def\C{{\mathbf C}}
\def\tf{{\emph{t-f}}}


% Title.
% ------
\title{Finite-Length Discrete Transforms}

%
% Author and date.
% ---------------
\date{\today}
\author{Germ\'an G\'omez-Herrero, \url{http://germangh.com}}




\begin{document}
\maketitle

\textbf{NOTE:} This document is a brief summary (a cheatsheet, actually) of Chapter 5 from the textbook \emph{Digital Signal Processing}, by S. K. Mitra. 



\begin{itemize}
\item The N-point DFT of a sequence $x[n]$ is defined as:

\begin{equation} \label{DFT}
X[k] = \sum_{n=0}^{N-1}x[n]e^{-j\frac{2\pi}{N}kn}=\sum_{n=0}^{N-1}x[n]W_{N}^{kn} \qquad 0\leq k \leq N-1
\end{equation}

where $W_{N}=e^{-j\frac{2\pi}{N}}$

\item The N-point inverse DFT (IDFT) of a sequence $X[k]$ is defined as:


\begin{equation} \label{IDFT}
x[n] = \frac{1}{N}\sum_{k=0}^{N-1}X[k]W_{N}^{-kn} \qquad 0\leq n \leq N-1
\end{equation}

\item The N-point DFT of a sequence $x[n]$ of length $L<N$ can be obtained by zero-padding the sequence $x[n]$ until it has the desired length of N samples. That is:

\[
\textrm{DFT}_{N}\{x[n]\}=\textrm{DFT}_{N}\{x_{zp}[n]\}
\]

where

\[
x_{zp}[n]=\left\{
\begin{array}{ll}
x[n] & \textrm{if} \quad 0 \leq n \leq L-1\\
0 & \textrm{if} \quad L \leq n \leq N\\ 
\end{array}
\right.
\]

\item Whenever computing DTFs or IDFTs it is useful to remember that:

\begin{equation}
\sum_{n=0}^{N-1}W_{N}^{An}=\left\{
\begin{array}{ll}
N & \textrm{if} \quad A = rN \quad \textrm{with} \quad r \quad \textrm{an integer}\\
0 & \textrm{otherwise}\\
\end{array}
\right.
\end{equation}


\item The DFT can also be expressed in matrix form like:

\begin{equation}
\X = \D_{N}\x
\end{equation}

where 

\[
\begin{array}{lll}
\X &=& \left[X[0],X[1],...,X[N-1]\right]^{T}\\
\x &=& \left[x[0],x[1],...,x[N-1]\right]^{T}\\
\end{array}
\]

and $\D_{N}$ is the $N\times N$ DFT matrix given by:

\begin{equation}
\D_{N}=\left[ 
\begin{array}{lllll}
1 & 1 & 1 & 1 & 1\\
1 & W_{N}^{1}& W_{N}^{2} & \cdots & W_{N}^{(N-1)} \\ 
1 & W_{N}^{2}& W_{N}^{4} & \cdots & W_{N}^{2(N-1)} \\
\vdots & \vdots & \vdots & \ddots & \vdots\\
1 & W_{N}^{(N-1)}& W_{N}^{2(N-1)} & \cdots & W_{N}^{(N-1)^{2}} \\
\end{array}
\right]
\end{equation}


\item Equivalently, the inverse DFT (IDFT) can be expressed in matrix form:

\begin{equation}
\x = \D_{N}^{-1}\X
\end{equation}

where:

\begin{equation}
\D_{N}^{-1}=\frac{1}{N}\D_{N}^{*}=\frac{1}{N}\left[ 
\begin{array}{lllll}
1 & 1 & 1 & 1 & 1\\
1 & W_{N}^{-1}& W_{N}^{-2} & \cdots & W_{N}^{-(N-1)} \\ 
1 & W_{N}^{-2}& W_{N}^{-4} & \cdots & W_{N}^{-2(N-1)} \\
\vdots & \vdots & \vdots & \ddots & \vdots\\
1 & W_{N}^{-(N-1)}& W_{N}^{-2(N-1)} & \cdots & W_{N}^{-(N-1)^{2}} \\
\end{array}
\right]
\end{equation}

\item An important operator when working with DTFs is the \emph{modulo} operator. This operator is denoted by $<a>_{b}=c$, which reads as ''$c$ \emph{is equal to} $a$ \emph{modulo} $b$''. This operator works differently depending whether $a$ is a positive or negative integer.

\[
<a>_{b}=\left\{
\begin{array}{ll}
a-\left\lfloor\frac{a}{b}\right\rfloor b & \textrm{if} \quad a>0\\
\left\lceil\frac{|a|}{b}\right\rceil b + a & \textrm{if} \quad a<0\\
\end{array}
\right.
\]

where $\left\lfloor\cdot\right\rfloor$ and $\left\lceil\cdot\right\rceil$ denote the \emph{floor} and \emph{ceiling} functions, respectively. For instance:

\[
\begin{array}{lll}
<10>_{4} &=&10-\left\lfloor\frac{10}{4}\right\rfloor \cdot 4 = 10 - 2\cdot 4 = 2\\
<-5>_{3} &=& \left\lceil\frac{|-5|}{3}\right\rceil\cdot 3 + (-5) = 2\cdot 3 + (-5) = 1\\
\end{array}
\]

\item The \emph{circular} convolution can be defined using the \emph{modulo} operator:

\begin{equation}
g[n]\stackrel{N}{\otimes}h[n]=\sum_{m=0}^{N-1}g[m]h[<n-m>_{N}] \qquad 0\leq n\leq N-1
\end{equation}

\item The DFT has several symmetry properties that can greatly simplify the computation of the DFT:

\vspace{1cm}
\begin{tabular}{cc}
\hline
Length-N sequence & N-point DFT\\
\hline
$x[n]$ & $X[k]$\\
$x^{*}[n]$ & $X^*\left[<-k>_{N}\right]$\\
Re$\{x[n]\}$& $X_{ccs}[k]=\frac{1}{2}\left[X\left[k\right]+X^*\left[<-k>_{N}\right]\right]$\\
$j\cdot \textrm{Im}\{x[n]\}$& $X_{cca}[k]=\frac{1}{2}\left[X\left[k\right]-X^*\left[<-k>_{N}\right]\right]$\\
$x_{ccs}[n]=\frac{1}{2}\left[x\left[n\right]+x^*\left[<-n>_{N}\right]\right]$&Re$\{X[k]\}$\\
$x_{cca}[n]=\frac{1}{2}\left[x\left[n\right]-x^*\left[<-n>_{N}\right]\right]$&$j\cdot\textrm{Im}\{X[k]\}$\\
\hline
\end{tabular}
\vspace{1cm}

\item Furthermore, if $x[n]$ is a length-N real sequence then the corresponding N-point DFT has the additional symmetry properties:

\[
\begin{array}{ccc}
X[k] &=& X^*[<-k>_{N}]\\
\textrm{Re}\{X[k]\}&=&\textrm{Re}\{X[<-k>_{N}]\}\\
\textrm{Im}\{X[k]\}&=&-\textrm{Im}\{X[<-k>_{N}]\}\\
\left|X[k]\right|&=&\left|X[<-k>_N]\right|\\
\textrm{arg}\{X[k]\}&=&-\textrm{arg}\{X[<-k>_{N}]\}\\
\end{array}
\]

\item The following theorems are also very useful when computing the DFT of a length-N sequence:

\vspace{1cm}
\begin{tabular}{cc}
\hline
Length-N sequence & N-point DFT\\
\hline
$g[n]$ & $G[k]$\\
$h[n]$ & $H[K]$\\
\hline
$\alpha g[n] + \beta h[n]$ & $\alpha G[k] + \beta H[k]$\\
$g[<n-n_{0}>_{N}]$ & $W_{N}^{kn_{0}}G[k]$\\
$W_{N}^{-k_{0}n}g[n]$ & $G[<k-k_0>_N]$\\
$g[n]\stackrel{N}{\otimes}h[n]$ & $G[k]H[k]$\\
$g[n]h[n]$ & $\frac{1}{N}G[k]\stackrel{N}{\otimes}H[k]$\\
\hline
\end{tabular}
\vspace{1cm}

\item The Parseval's relation can be also useful, specially when we want to compute the energy of a sequence $g[n]$ that has a DFT $G[k]$:

\[
\sum_{n=0}^{N-1}|g[n]|^2=\frac{1}{N}\sum_{k=0}^{N-1}|G[k]|^2
\]

\item The circular convolution of two sequences $g[n]$ and $h[n]$ can be computed using the DFT:

\[
\left.
\begin{array}{lll}
g[n]&\stackrel{DFT}{\rightarrow}&G[k]\\
h[n]&\stackrel{DFT}{\rightarrow}&H[k]\\
\end{array}
\right\}
\Longrightarrow g[n]\stackrel{N}{\otimes}h[n]=\textrm{IDFT}\left\{G[k]H[k]\right\}
\]


\item The linear convolution of a length-$L_{x}$ sequence $x[n]$ and a length-$L_y$ sequence $y[n]$ can also be computed using the DFT:

\[
x[n]\otimes y[n]=x[n]\stackrel{N}{\otimes}y[n]=\textrm{IDFT}\left\{\textrm{DFT}_{N}\left\{x[n]\right\}\textrm{DFT}_{N}\left\{x[n]\right\}\right\}
\]

where $N=L_x+L_y-1$.

\item The DFTs of two length-$N$ real sequences $g[n]$ and $h[n]$ can be computed using the DFT of a single complex sequence $x[n]=g[n]+jh[n]$:

\[
\left.
\begin{array}{lll}
g[n]&\stackrel{DFT}{\rightarrow}&G[k]\\
h[n]&\stackrel{DFT}{\rightarrow}&H[k]\\
x[n]&\stackrel{DFT}{\rightarrow}&X[k]\\
\end{array}
\right\}
\Longrightarrow 
\left\{
\begin{array}{lll}
G[k] &=& \frac{1}{2}\left[X[k]+X^*[<-k>_{N}]\right]\\
H[k] &=& \frac{1}{2j}\left[X[k]-X^*[<-k>_{N}]\right]\\
\end{array}
\right.
\]


\end{itemize}




%\begin{figure}[h!]
%\centering
%\includegraphics[width=.5\textwidth]{fig3.eps}
%\caption{Two ways of cascading and up-sampler and a down-sampler.}
%\label{fig3}
%\end{figure}




\end{document}
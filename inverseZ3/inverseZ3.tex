\documentclass[a4paper,11pt,oneside]{article}

\usepackage{amsmath,amssymb,epsfig}
\usepackage[T1]{fontenc}
\usepackage{ae,aecompl}
\usepackage{url}
\usepackage{subfigure}
%\addtolength{\voffset}{-.5cm}
%\addtolength{\hoffset}{-.5cm}
\setlength{\parindent}{0in}
%\addtolength{\textwidth}{.5cm}
%\addtolength{\textheight}{.51cm}
\addtolength{\parskip}{.5cm}

% Example definitions.
% --------------------
\def\x{{\mathbf x}}
\def\X{{\mathbf X}}
\def\u{{\mathbf u}}
\def\U{{\mathbf U}}
\def\x{{\mathbf x}}
\def\s{{\mathbf s}}
\def\A{{\mathbf A}}
\def\y{{\mathbf y}}
\def\W{{\mathbf W}}
\def\w{{\mathbf w}}
\def\B{{\mathbf B}}
\def\D{{\mathbf D}}
\def\a{{\mathbf a}}
\def\D{{\mathbf D}}
\def\P{{\mathbf P}}
\def\n{{\mathbf n}}
\def\V{{\mathbf V}}
\def\R{{\mathbf R}}
\def\I{{\mathbf I}}
\def\M{{\mathbf M}}
\def\sech{{\mathrm{sech}}}
\def\L{{\cal L}}
\def\Cum{{\rm{Cum}}}
\def\var{{\rm{var}}}
\def\T{{\mathbf T}}
\def\C{{\mathbf C}}
\def\tf{{\emph{t-f}}}


% Title.
% ------
\title{Inverse Z transform: Example 3 (directly invertible)}
%
% Author and date.
% ---------------
\date{\today}
\author{Germ\'an G\'omez-Herrero, \url{http://germangh.com}}



\begin{document}
\maketitle

Given the following system function of a causal system:

\begin{equation}\label{eq:Yz}
H(z)=\frac{-5-3z+2z^{-1}}{1-2z^{-1}} \qquad \textrm{ROC}\equiv |z|>2 
\end{equation}

Find the impulse response $h[n]$ of the system.


\vspace{1cm}

\textbf{SOLUTION:}

Although the numerator of $H(z)$ is a polynomial of greater degree than the denominator you do not need to perform long division. You should notice that this Z-transform is already directly invertible using the shifting property of the Z-transform (without needing to compute residuals or long division):

\[
H(z)=-5\underbrace{\frac{1}{1-2z^{-1}}}_{G(z)}-3z\underbrace{\frac{1}{1-2z^{-1}}}_{G(z)}+2z^{-1}\underbrace{\frac{1}{1-2z^{-1}}}_{G(z)}
\]

So we finally obtain that:

\begin{equation}\label{eq:yn}
h[n] = -5(2)^n\mu[n]-3(2)^{n+1}\mu[n+1]+2(2)^{n-1}\mu[n-1]
\end{equation}

\end{document}